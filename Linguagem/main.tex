\documentclass{article}
\usepackage[utf8]{inputenc}
\usepackage[portuguese]{babel}

\usepackage{listings}
\usepackage{color}
\usepackage{xcolor}
\usepackage{url}
 
\definecolor{codegreen}{rgb}{0,0.6,0}
\definecolor{codegray}{rgb}{0.5,0.5,0.5}
\definecolor{codepurple}{rgb}{0.58,0,0.82}
\definecolor{backcolour}{rgb}{0.95,0.95,0.92}
 
\lstdefinestyle{slstyle}{
    backgroundcolor=\color{backcolour},   
    commentstyle=\color{codegreen},
    keywordstyle=\color{magenta},
    numberstyle=\tiny\color{codegray},
    stringstyle=\color{codepurple},
    basicstyle=\footnotesize,
    language=C,
    morekeywords={*, sequence, instrument, performance, in, on, play, always, at, and, except, after, sequentially, times, loop, number},
    breakatwhitespace=false,         
    breaklines=true,                 
    captionpos=t,                    
    keepspaces=true,                 
    numbers=left,                    
    numbersep=5pt,                  
    showspaces=false,                
    showstringspaces=false,
    showtabs=false,                  
    tabsize=2
}
\lstset{style=slstyle}

\usepackage{caption}
\DeclareCaptionFont{white}{\color{white}}
\DeclareCaptionFormat{listing}{\colorbox{gray}{\parbox{\textwidth}{#3}}}
\captionsetup[lstlisting]{format=listing,labelfont=white,textfont=white}

\title{Especificação da Linguagem}
\author{Grupo 12, LFA}
\date{Junho 2018}

\usepackage{natbib}
\usepackage{graphicx}

\begin{document}

\maketitle

\tableofcontents
\vspace{2cm} %Add a 2cm space

\section{Estruturas de dados (e tipos de variáveis)} \label{variables}
\subsection{Sequências}
Uma sequência de notas e silêncios pode ser definida, através da palavra chave \texttt{sequence}, como:
\begin{lstlisting} 
sequence melody = [C C G G A A G R F F E E D D C R]; // os espacos sao opcionais
\end{lstlisting}

\subsubsection{Notas musicais}
A notas são identificadas pela sua letra:
\begin{itemize}
    \item \texttt{C} para Dó,
    \item \texttt{D} para Ré,
    \item \texttt{E} para Mi,
    \item \texttt{F} para Fá,
    \item \texttt{G} para Sol,
    \item \texttt{A} para Lá,
    \item \texttt{B} para Si.
\end{itemize}
A letra \texttt{R} é usada para silêncios (derivada de \textit{Rests}).

Um cardinal (\texttt{\#}) a seguir à letra sobe a respetiva nota por meio tom; um b minúsculo (\texttt{b}) a seguir à letra desce a respetiva nota por meio tom. Mais que um \texttt{\#} ou \texttt{b} podem ser aplicados a uma nota (por exemplo, \texttt{C\#\#}).

Pode, depois do tom, aparecer um número, entre 0 e 8. Este especifica a oitava a usar, sendo 0 a mais grave e 8 a mais aguda. Por omissão, a quarta oitava é usada.

Assim, a sequência já apresentada podia reescrita da seguinte forma:
\begin{lstlisting} 
sequence melody = [C4 B# G G4 A4 A G R F E#4 Fb4 E D D4 C4 R];
\end{lstlisting}

Finalmente, uma sequência pode ser definida a partir doutra sequência:
\begin{lstlisting} 
sequence melody = [C C G G A A G R F F E E D D C R];

sequence r = [E G E melody D E D C]; // so e possivel utilizar esta nomenclatura se 'melody' for uma sequence
// equivalente a:
// sequence r = [EGE CCGGAAGR FFEEDDCR DEDC];
\end{lstlisting}


\subsubsection{Duração duma nota}
Por omissão, uma nota demora um tempo\footnote{Em termos musicais, uma semínima.}. A sua duração depende do \textit{tempo}, ou \texttt{BPM}\footnote{\textit{Beats Per Minute}}, da música. Esta configuração é explorada na secção \ref{config}.

Uma nota pode, no entanto, tocar mais ou menos tempo. Há duas formas de especificar a duração duma dada nota:
\begin{enumerate}
    \item \textbf{Por extensão.} A duração da nota é especificada através de chavetas (\texttt{\{} e \texttt{\}}), relativamente à duração unitária. Por exemplo, \texttt{C\{4\}} demora o quádruplo do tempo de \texttt{C}, e \texttt{C\{0.75\}} demora três quartos do tempo de \texttt{C}.
    \item \textbf{Notação simplificada.} Utilizam-se apóstrofos para reduzir a duração duma nota em metade, havendo uma correspondência direta com a duração das notas musicais convencionadas
    \footnote{Semínima (duração de 1), colcheia (duração de 1/2), semicolcheia (duração de 1/4), etc.}. 
    Por exemplo, \texttt{C'} demora metade do tempo de \texttt{C}, e \texttt{C''} apenas um quarto. 
    % suportar aumentar tempos em notacao simplificada?
\end{enumerate}

\subsubsection{Acordes}
O símbolo \texttt{|} é usado para tocar várias notas em simultâneo, numa só sequência.
Para tocar o acorde de Dó maior (C, E e G), na quarta oitava, podíamos então escrever:
\begin{lstlisting} 
sequence intro = [C|E|G];
\end{lstlisting}

\subsection{Performances}
A associação duma sequência musical com um instrumento representa um terceiro tipo de dados, uma \texttt{performance}:
\begin{lstlisting} 
sequence twinkle = [CC GG AA G{2} FF EE DD C{2}];
performance p = twinkle on guitar;

// ou, alternativamente, definindo a sequencia implicitamente
performance p = [CC GG AA G{2} FF EE DD C{2}] on guitar;
\end{lstlisting}

\subsection{Números}
Números (inteiros ou reais) são também suportados:
\begin{lstlisting} 
number num = 4;
\end{lstlisting}
Números podem também ser usados para referenciar um dado momento entre o início e o fim, inclusive, da peça musical. O início da peça é representado pelo número 0. 
\subsubsection{Operador de duração}
O operador \texttt{|}\textit{x}\texttt{|}, aceita um parâmetro (\textit{x}), do tipo sequence ou performance, e devolve um número igual à sua duração.
\begin{lstlisting} 
sequence intro = [R{4} C{4} G{4} C5{3.5} E|G|C5{.5} Eb|G|C5{8} C{4} G{4}]; // Strauss - Also Sprach Zarathustra - Intro (https://www.8notes.com/scores/7213.asp)
number endIntro = |intro|;
\end{lstlisting}

\subsection{Instrumentos}
Para tocar sequência musical é, naturalmente, necessário especificar que instrumento deve ser utilizado. Assim, para tocar \textit{Twinkle, Twinkle, Little Star} com uma piano, teríamos:
\begin{lstlisting} 
// definir a sequencia
sequence twinkle =  [CC GG AA G{2} FF EE DD C{2}];

// utilizando performances
performance p = twinkle on piano;
\end{lstlisting}

Vários instrumentos podem ser usados para tocar uma dada sequência. No entanto, nem todos suportam o mesmo registo (por exemplo, um piano suporta uma gama maior de notas que um violino). Se a um instrumento é dada uma sequência que este não suporta, as notas não suportadas são substituídas pela nota mais próxima que é suportada. % doable?

Os instrumentos disponíveis são os seguintes:
\begin{itemize}
    \item \texttt{piano};
    \item \texttt{guitar};
    \item \texttt{violin};
    \item \texttt{cello};
    \item \texttt{bass};
    \item \texttt{drums}.
\end{itemize}
A criação de novos instrumentos é suportada, num ficheiro externo de tipo auxiliar, estando detalhada na secção \ref{config}. No ficheiro de tipo principal, não é possível definir ou redefinir novos instrumentos, sendo apenas possível usá-los como constantes pré-definidas.

\subsection{Arrays}
Um array é uma coleção de várias instâncias da mesma estrutura de dados. Suportam-se arrays de sequências, instrumentos, performances e números.

Um array pode ser definido de duas formas distintas:
\begin{lstlisting} 
// criar um array com 4 instrumentos - e necessario usar a palavra chave instrument para indicar que e um array de instrumentos
instrument[] band = [piano, guitar, bass, drums];

// outra forma de definir o mesmo array
instrument[] band = piano and guitar and bass and drums;

// criar um array com 3 performances
performance[] p = [
    [D{1.5} D{0.5}   E  D G F#{2}] on piano, 
    [D{1.5} D{0.5}   E  D A G{2}] on bass,
    [D{1.5} D{0.5}   D5 B G F# E] on guitar];
\end{lstlisting} 

Pode usar-se um array para criar outro. A palavra chave \texttt{and} anexa ao fim do array já existente o elemento ou elementos dados.  A palavra chave \texttt{except} retira o elemento ou elementos dados do array, se lá estiverem presentes.

\begin{lstlisting} 
instrument[] band = [piano, guitar, bass, drums];

// definir arrrays a partir de outros arrays
instrument[] new_band = band and violin;
instrument[] left_over_band = band except bass;
\end{lstlisting} 

Para arrays de números, o operador \texttt{a->b}, que devolve um array de inteiros de tipo \texttt{[a, a+1, ..., b-1, b]} pode também ser utilizado para gerar um array.
\begin{lstlisting}
start_times = 0->3;
// equivalente a:
//    start_times = [0, 1, 2, 3];
\end{lstlisting}

Para se aceder a um instrumento, a notação de parênteses retos, começando a contar no 0, é usada. A notação de intervalo, com dois pontos (\texttt{:}) é também suportada.
\begin{lstlisting} 
performance[] p = [
    [D{1.5} D{0.5}   E  D G F#{2}] on piano, 
    [D{1.5} D{0.5}   E  D A G{2}] on bass,
    [D{1.5} D{0.5}   D5 B G F# E] on guitar];
    
// aceder ao primeiro elemento
performance intro = p[0];

// exemplo da notacao de intervalo
performance[] q = p[0:1] and [C5{1.5} C5{0.5} B G A G{3}] on violin and p[2:];
\end{lstlisting}


\section{Geração de aúdio} \label{audio}
\subsection{Reprodução}
Para reproduzir uma performance, utiliza-se a palavra chave \texttt{play}:
\begin{lstlisting} 
// definir uma performance
sequence twinkle =  [CC GG AA G{2} FF EE DD C{2}];
performance p = twinkle on piano;

// reproduzir a performance
play p;

// alternativamente, definir a performance implicitamente
play twinkle on piano;

// ou definir a performance e a sequencia implicitamente
play [CC GG AA G{2} FF EE DD C{2}] on piano;
\end{lstlisting} 
\subsection{Modos de reprodução}
No exemplo anterior, não foi especificado quando começar a tocar a sequência. Por omissão, a sequência começa a ser tocada no início da peça (por outras palavras, no tempo 0).

Averiguemos os diferentes modos de reprodução:

\subsubsection{Simultâneo}
Por omissão, todas as sequências são tocadas começando no tempo 0. Se há mais que uma sequência a ser tocada, todas as sequências são tocadas em paralelo.

\begin{lstlisting} 
play [CC GG AA G{2} FF EE DD C{2}] on piano;

// e equivalente, em termos do som produzido no ficheiro final, a
play [CC RR RR G{2} FR RE DR C{2}] on piano;
play [RR GG AA RR RF ER RD RR] on piano;
\end{lstlisting}

Um outro exemplo de reprodução simultânea é o seguinte, que separa melodia e harmonia em duas performances diferentes:
\begin{lstlisting} 
// definir sequencias
sequence twinkle = [CC GG AA G{2} FF EE DD C{2}];
sequence twinkle_bass = [C|E|G{2} F|A|C{2} C|E|G{4} F|A|C C|E|G G|B|D C|E|G];

// tocar performances
play twinkle on guitar;
play twinkle_bass on guitar;
\end{lstlisting}

\subsubsection{Sequencial}
A palavra chave \texttt{after} indica que uma dada performance deve começar imediatamente após o fim doutra.
\begin{lstlisting} 
// definir sequencias
sequence first_line = [CC GG AA G{2}];
sequence second_line = [FF EE DD C{2}];

// definir performances
performance first_line = twinkle on guitar;
performance second_line = twinkle_bass on guitar;

// tocar performances
play first_line;
after first_line play second_line;
\end{lstlisting}

Num exemplo mais avançado, onde a sequência de referência (no exemplo acima, \texttt{first\_line}) passada a after é tocada mais que uma vez, pode ser especificado após que performances deve a sequência alvo (no exemplo acima, \texttt{second\_line}) ser tocada. Por omissão, a sequência alvo é tocada apenas 1 vez, após a primeira reprodução da sequência de referência. 

Para obter outros comportamentos, a palavra chave \texttt{always} pode ser utilizada.

\begin{lstlisting} 
// definir sequencias
sequence first_line = [CC GG AA G{2}];
sequence second_line = [FF EE DD C{2}];

// definir performances
performance first_line = twinkle on guitar;
performance second_line = twinkle_bass on guitar;

// tocar performances
play first_line;
after first_line always play second_line;
after second_line play first_line;
// toca first_line (FL), seguido de second_line (SL), e repete uma vez, ou seja, FL, SL, FL, SL
\end{lstlisting}

\subsubsection{Usando Números}
A palavra chave \texttt{at} indica um número específico no qual a performance deve começar, independentemente de haver outras performances a decorrer nesse momento. 
\begin{lstlisting} 
performance verse = [CC E{2} GG B{2} C5C5 GG C{4}] on violin;
performance chorus = [GAGA ABBA GEGE EBBE] on violin;

number chorusStart = |verse|;

// tocar performances
play verse;
at chorusStart play chorus;
\end{lstlisting}

Uma performance pode ser tocada em mais que um momento.
\begin{lstlisting} 
performance verse = [CC E{2} GG B{2} C5C5 GG C{4}] on violin;
performance chorus = [GAGA ABBA GEGE EBBE] on violin;

number chorusStart = 2*|verse|;
number otherTimeChorusStarts = 3.14*|verse|;

// tocar performances (chorus e tocada 2 vezes)
play verse;
at chorusStart play chorus;
at otherTimeChorusStarts play chorus;
\end{lstlisting}

\subsubsection{Repetição}
Há duas palavras chaves que permitem a repetição: o uso da palavra chave \texttt{times} permite repetir uma performance 0 ou mais vezes. \texttt{loop} permite repetir uma performance até ao fim da peça (determinado pelo término da sequência que acabe mais tarde).
\begin{lstlisting} 
performance verse = [CC E{2} GG B{2} C5C5 GG C{4}] on violin;
performance chorus = [GAGA ABBA GEGE EBBE] on violin;
performance bass = [G|B|D{4} A|D|F#{4} G|B|D{4} A|D|F#{4}] on bass;

number chorusStart = 2*|verse|;

// tocar performances
play verse 2 times on piano;
at chorusStart play chorus;

loop bass; // repete ate ao fim da musica
\end{lstlisting}

\subsection{Uso de arrays}
Um array pode ser reproduzido de forma simultânea (forma utilizada por omissão) ou sequencialmente (através do uso da palavra chave \texttt{sequentially}).

\begin{lstlisting} 
// exemplo com sequencias
sequence[] melody_lines = [
    [D{1.5} D{0.5}   E  D G F#{2}], 
    [D{1.5} D{0.5}   E  D A G{2}],
    [D{1.5} D{0.5}   D5 B G F# E],
    [C5{1.5} C5{0.5} B  G A G{3}]];

play melody_lines /*sequentially*/ on piano; // descomentar sequentially para tocar melody_lines de forma sequencial
\end{lstlisting}

\begin{lstlisting} 
// exemplo com instrumentos
instrument[] band = [piano, guitar, bass, drums];

play melody_lines[0] /*sequentially*/ on band;
\end{lstlisting}

\begin{lstlisting} 
// exemplo com performances
performance[] perfors = [
    [D{1.5} D{0.5}   E  D G F#{2}] on piano, 
    [D{1.5} D{0.5}   E  D A G{2}] on bass,
    [D{1.5} D{0.5}   D5 B G F# E] on guitar];
    
play perfors /*sequentially*/;
\end{lstlisting}
No entanto, não é possível reproduzir um array de sequências num array de instrumentos. Isto é, por exemplo, não é possível fazer o seguinte:

\begin{lstlisting} 
// definir array de sequencias
sequence[] melody_lines = [
    [D{1.5} D{0.5}   E  D G F#{2}], 
    [D{1.5} D{0.5}   E  D A G{2}],
    [D{1.5} D{0.5}   D5 B G F# E],
    [C5{1.5} C5{0.5} B  G A G{3}]];

// definir array de instrumentos
instrument[] band = [piano, guitar, bass, drums];

play melody_lines/*sequentially*/ on band; // ilegal!!
\end{lstlisting}

Para se obter este tipo de resultados, têm que ser utilizadas instruções de repetição (ver secção \ref{flux}).

% (AFTER _ (ALWAYS?) | AT _ (AND _)* )? PLAY _ SEQUENTIALLY? (INT TIMES)? ON INST;

Arrays de números não podem ser usados diretamente na instrução \texttt{play}. No entanto, o mesmo resultado final pode ser obtido usando instruções de repetição, exploradas na secção \ref{flux}.

\subsection{Modulações}
Podem obter-se versões modificadas de sequências ou performances através dos operadores de modulação. Estes operadores devolvem uma nova sequência ou performance, alterada em algum aspeto (tom ou tempo) em relação a uma dada sequência  ou performance original, respetivamente.
\subsubsection{de Tom}
Pode mudar-se o tom de uma dada sequência ou performance através dos operadores \texttt{+} e \texttt{-}. A sequência ou performance devolvida será a sequência ou performance original com todas as suas notas subidas ou descidas \textit{n} meios-tons\footnote{Doze meios-tons constituem uma oitava.}.
\begin{lstlisting} 
// sequencia original
sequence s = [D{1.5} D{0.5} E D G F#{2}];

sequence s1 = s - 36; // diminuir a oitava por 3 (36 = 3*12)
// equivalente a dizer:
//    sequence s1 =  [D1{1.5} D1{0.5} E1 D1 G1 F#1{2}];

// mudar oitava da sequencia
sequence s5 = s + 12; // aumentar a oitava por 1 (12 = 1*12)
// equivalente a dizer:
//    sequence s5 =  [D5{1.5} D5{0.5} E5 D5 G5 F#5{2}];
\end{lstlisting}
\begin{lstlisting} 
// performance original
performance p = [D{1.5} D{0.5} E D G F#{2}] on bass;

play p+5;
// equivalente a:
//    play  [F#{1.5} F#{0.5} G# F# B A#{2}] on bass;
\end{lstlisting}

\subsubsection{de Tempo}
Pode mudar-se o tempo (ou seja, a velocidade) de um dada sequência ou performance através dos operadores \texttt{*} e \texttt{/}. A sequência ou performance devolvida será a sequência ou performance original com todas as suas notas aceleradas ou atrasadas \textit{n} vezes.\footnote{A nova velocidade é dada por \texttt{velocidade original * fator}, ou por \texttt{velocidade original / fator}, conforme os operadores \texttt{*} ou \texttt{/} são usados, respetivamente.}
\begin{lstlisting} 
// performance original
performance p = [D{1.5} D{0.5} E D G F#{2}] on bass;

play p * 5; // toca a sequencia 5x mais rapido (cada nota dura 1/5 do seu tempo original)
// equivalente a:
//    play  [D{.3} D{0.1} E{.2} D{.2} G{.2} F#{.4}] on bass;
\end{lstlisting}

\begin{lstlisting} 
// performance original
performance p = [D{1.5} D{0.5} E D G F#{2}] on bass;

play (p + 5) * 5; // toca a sequencia 5x mais rapido, com todas as notas 5 meios-tons mais agudas
// equivalente a:
//    play  [F#{.3} F#{0.1} G#{.2} F#{.2} B{.2} A#{.4}] on bass;
\end{lstlisting}

\section{Interação com o exterior} \label{exterior}
\subsection{Estruturas de dados auxiliares}
\subsubsection{Strings}
Strings são sequências de caracteres, números e símbolos delimitadas por aspas (\texttt{"}). Dentro duma string, aspas podem ser escapadas através de \texttt{\textbackslash"}.

Não existe um tipo de dados String explícito, sendo este usado apenas como parâmetro opcional para funções de I/O.
\subsection{getInt( string? )}
\texttt{getInt()} permite obter um inteiro através do \textit{Standard In}. 

Opcionalmente, pode ser passada uma String, que será impressa antes de aguardar a resposta do utilizador (uma String de \textit{prompt}).


\section{Controlo de fluxo} \label{flux}
\subsection{Instruções condicionais}
\subsubsection{if}
As palavras chave \texttt{if}, \texttt{else if} e \texttt{else} permitem testar condições. Os operadores suportados numa condição são os de igualdade (\texttt{==}), desigualdade (\texttt{!=}), menor (\texttt{<}), maior (\texttt{>}), menor ou igual (\texttt{<=}), e maior ou igual (\texttt{>=}).
\begin{lstlisting} 
sequence s = getSequence("Enter a sequence: ");

if (|s| > 5) {
    play s on piano;
} else if (|s| > 2) {
    play s on cello;
} else {
    play s on guitar;
}
\end{lstlisting}

\subsection{Intruções de repetição}
\subsubsection{for}
As palavras chave \texttt{for} e \texttt{in} permitem definir instruções de repetição, ou seja, permitem que um dado código seja executado múltiplas vezes, iterando sobre todos os elementos de um dado array.
\begin{lstlisting} 
// instrumentos 
instrument[] band = [piano, guitar, bass, drums];
for instrument inst in band {
    play [ABCDCBA] on inst;
}

// sequencias
sequence[] sequences = [
    [D{1.5} D{0.5}   E  D G F#{2}], 
    [D{1.5} D{0.5}   E  D A G{2}],
    [D{1.5} D{0.5}   D5 B G F# E],
    [C5{1.5} C5{0.5} B  G A G{3}]];
    
for sequence seq in sequences {
    play seq on piano;
}
    
// performances
performance[] perfors = [
    [D{1.5} D{0.5}   E  D G F#{2}] on piano, 
    [D{1.5} D{0.5}   E  D A G{2}] on bass,
    [D{1.5} D{0.5}   D5 B G F# E] on guitar];
    
number t = 0;
for performance perfor in perfors {
    at t play perfor;
    t = t + |perfor|;
}

// numeros
start_times = [0, 1, 3, 7];
for number t in start_times {
    at t play [C1 E1 G1 E1] on piano;
}
\end{lstlisting}

Caso se pretenda iterar sobre algum código um dado número de vezes, sem haver correspondência direta entre esse número e o conteúdo de um dado array, pode usar-se números num dado intervalo, usando o operador \texttt{a->b} (que devolve um array de inteiros de tipo \texttt{[a, a+1, ..., b-1, b]}):
\begin{lstlisting} 
instrument[] band = [piano, guitar, bass, drums];
sequence[] sequences = [
    [D{1.5} D{0.5}   E  D G F#{2}], 
    [D{1.5} D{0.5}   E  D A G{2}],
    [D{1.5} D{0.5}   D5 B G F# E],
    [C5{1.5} C5{0.5} B  G A G{3}]];

number t = 0;
for number i in 0->3 {
    at t play sequences[i] on band[i];
    t = t + |sequences[i]|;
}
\end{lstlisting}

\section{Configurações (ficheiro auxiliar)} \label{config}
Um ficheiro do tipo principal (ficheiros com extensão \texttt{.principal}) suporta um todas as operações descritas até este ponto. 
Várias configurações podem ser feitas no ficheiro auxiliar (ficheiros com extensão \texttt{.auxiliar}). Nesta secção, vamos abordar as diferentes configurações que podem ser definidas através do ficheiro auxiliar.
\subsection{BPM (Beats Per Minute)}
\texttt{BPM} é uma palavra reservada\footnote{BPM não pode ser usado como nome de uma variável.} usada para configurar o \textit{tempo} da música. A configuração deve ser feita no ficheiro de configuração, mas é possível reescrevê-la no ficheiro principal (o que define a música a gerar).
\begin{lstlisting} 
BPM = 160;
\end{lstlisting}

\subsection{Especificação de (novos) instrumentos}
O formato \textit{midi} suporta 128 instrumentos diferentes\footnote{Para obter mais informação sobre os diferentes instrumentos disponíveis, ver \url{http://www.ccarh.org/courses/253/handout/gminstruments/}.}.

É possível criar novos instrumentos associando a uma palavra um número, através da palavra chave \texttt{instrument}. O nome dado ao instrumento deixa de poder ser usado como nome de variável.

\begin{lstlisting} 
// definir dois novos instrumentos
instrument strings: 49;
instrument synth1: 81;
\end{lstlisting}

Os instrumentos já definidos estão associados aos seguintes códigos: 1(\texttt{piano}), 25(\texttt{guitar}), 41(\texttt{violin}), 43(\texttt{cello}), 44(\texttt{bass}), e 119(\texttt{drums}). 

É também possível definir novos instrumentos à custa de instrumentos já existentes, e associar nomes a tons.

\begin{lstlisting} 
// duplicar um instrumento ja existente
instrument percussion: drums;

// mapear tons (associar nomes a tons)
AcousticBassDrum = B0;
BassDrum1 = C1;
SideStick = C#1;
AcousticSnare = D1;
HandClap = D#1;
ElectricSnare = E1;
LowFloorTom = F1;
ClosedHiHat = F#1;
HighFloorTom = G1;
PedalHiHat = G#1;
LowTom = A1;
OpenHiHat = A#1;
LowMidTom = B1;
HiMidTom = C2;

// a um tom podem ser associados varios nomes
// (mas a um tom nao podem ser associadas mais que uma nota)
LongWhistle = C4;
MiddleC = C4;
                     	 
\end{lstlisting}	 
É ainda possível criar um instrumento juntando vários instrumentos já existentes. Associa-se a uma nota (ou a um conjunto de notas, sendo o intervalo representado por \texttt{NotaInicial - NotaFinal}, incluindo os extremos) o instrumento que deve ser utilizado para a tocar, utilizando o operador \texttt{->}.

Se uma nota for definida mais que uma vez, a última definição será a usada. Se uma nota não for definida, gerará um erro se for reproduzida.
\begin{lstlisting} 
// definir um novo instrumento a partir de instrumentos existentes
instrument thing: B0-C4 -> drums,
                  G4 -> guitar,
                  A4-G5 -> guitar, 
                  C5-A8 -> piano; // notas C5-G5 redifinidas
\end{lstlisting}
% 49	C#2	Crash Cymbal 1	
% 50	D2	High Tom
% 51	Eb2	Ride Cymbal 1
% 52	E2	Chinese Cymbal	
% 53	F2	Ride Bell	
% 54	F#2	Tambourine	
% 55	G2	Splash Cymbal
% 56	Ab2	Cowbell	
% 57	A2	Crash Cymbal 2	
% 58	Bb2	Vibraslap	 
% 59	B2	Ride Cymbal 2
% 60	C3	Hi Bongo	
% 61	C#3	Low Bongo
% 62	D3	Mute Hi Conga
% 63	Eb3	Open Hi Conga
% 64	E3	Low Conga	
% 65	F3	High Timbale	
% 66	F#3	Low Timbale
% 67	G3	High Agogo	
% 68	Ab3	Low Agogo
% 69	A3	Cabasa
% 70	Bb3	Maracas
% 71	B3	Short Whistle
% 72	C4	Long Whistle
% 73	C#4	Short Guiro	
% 74	D4	Long Guiro	
% 75	Eb4	Claves
% 76	E4	Hi Wood Block
% 77	F4	Low Wood Block
% 78	F#4	Mute Cuica	
% 79	G4	Open Cuica
% 80	Ab4	Mute Triangle
% 81	A4	Open Triangle	


\section{Exemplos (TODO)} \label{example}
\subsection{Parabéns}
\begin{lstlisting}[caption=parabens.auxiliar]
// set default BPM setting
BPM = 160;
\end{lstlisting}

\begin{lstlisting}[caption=parabens.principal]
// get user input
number repeat_times = getInt("Number of repetitions: ");

// define melody sequences
sequence[] melody_lines = [
    [D{1.5} D{0.5}   E  D G F#{2}], 
    [D{1.5} D{0.5}   E  D A G{2}],
    [D{1.5} D{0.5}   D5 B G F# E],
    [C5{1.5} C5{0.5} B  G A G{3}]];

// perform
play melody_lines sequentially repeat_times times on piano;
\end{lstlisting}

\subsection{Looping machine like thingy?}
\begin{lstlisting} 
// tipo, add looping track, add looping track, etc.
\end{lstlisting}

\subsection{Mais exemplos}
\begin{lstlisting} 
\end{lstlisting}

% \bibliographystyle{plain}
% \bibliography{references}
\end{document}
